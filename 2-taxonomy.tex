\section{Bootstrapping Taxonomy}

Out-of-band (OOB) channels are auxiliary to the main frequency of communication and are used as a means to bolster security.
OOB channels are characterized by the uncommon or unused frequencies that the protocols use to communicate or share any cryptographic information.
This section identifies techniques that can be used in the design of onboarding protocols highlighting the security properties and vulnerabilities for each of them. 

\subsection{Bluetooth Low Energy (BLE)}
\subsubsection{Device Provision Protocol (DPP)}
Seamless operations in a smart office involves communications among devices which are separated by small distances.
In such scenarios, BLE acts as a viable option for device onboarding as against WiFi.
sThe functioning of DPP[6] involves two devices: a configurator which broadcasts intent and an enrollee that responds to a broadcast when within the range.
After a response to the broadcast, the devices move to an auxiliary channel via BLE and exchange cryptographic information.
DPP operation is dependent on the configurator sending out a request packet on the auxiliary channel and the enrollee replying with the same auxiliary channel request information to successfully onboard.
One of the main vulnerabilities of using BLE pairing is that it broadcasts messages openly to any device within range which makes them susceptible to Man in The Middle (MiTM) attacks.
BLE itself does not have a proof of possession of the bootstrapping keys in the auxiliary channel.
Any device within range can possibly eavesdrop on both device and obtain the bootstrapping keys.


%As a means for device onboarding, BLE is a viable alternative to Wifi in scenarios that do not require long range communication - not unlike an office. BLE lets a user send out a broadcast that states their intent to bootstrap with another device. The DPP requires two devices: a configurator that broadcasts intent and an enrollee within range that responds to a broadcast. When an enrollee does respond to a broadcast, the devices move to an auxiliary channel via BLE and exchange cryptographic information. One of the primary DPP characteristics is that the configurator sends out a request packet on the auxiliary channel and the enrollee must send back that same auxiliary channel request information to successfully onboard. One of the main vulnerabilities of using BLE pairing is its that broadcasts openly to any device within range. By broadcasting openly, the devices make themselves susceptible to Man in The Middle (MiTM) attacks. BLE itself does not have a proof of possession of the bootstrapping keys in the auxiliary channel. Any device within range can possibly eavesdrop on both device and obtain the bootstrapping keys.


\subsection{Buttons}
\subsubsection{Button Enabled Device Association (BEDA)}
The BEDA protocol[18] is reliant on physical buttons on two devices.
These buttons are integral to the protocol because it uses the duration of a pressed button which is used to generate a shared secret.
However, this method presents a few vulnerabilities to replay attacks because hashes are not encrypted when they are sent out which lets an attacker pose as the sender by sending the hash to the original receiving device.



\subsection{Magnetometer Pairing}
Magnetometer pairing [23] is technique of bootstrapping that was designed for smartphones by making use of their magnetometers and a WiFi connection. To initiate the process two devices must be tapped together which should trigger the protocol depending on the threshold that has been set. Once the process has been initiated the devices will go through a standard Diffie-Hellman (DH) Key Exchange. During this process the devices will record their magnetometer readings at the same time. If the procedure fails, the users will be asked to change their device orientation to reinitiate pairing.  In order to foil eavesdroppers, the protocol makes use of the Interlock protocol to bolster its security against passive attacks. The protocol is secure against MiTM attacks because the Interlock the attacker will only receive half of an encrypted message of which is has no key to decrypt. The protocol is secure against Denial of Service (DoS) attacks because it is extremely difficult to overload the magnetometer readings without an abnormally large magnet, in which case the attacker will make themselves known. Magnetometer pairing makes use of the randomness of correlated sensor data by incorporating the lack of temporal or spatial alignment between devices into the encrypted secret. This randomness comes from the strength of the magnetic field at certain position, ambient noise, and device orientation.

\subsection{Audio Pairing}
\subsubsection{Human-Assisted Pure Audio Device Pairing (HAPADEP)}
The HAPADEP protocol [12] makes use of the sonic frequency of sound waves by using two devices that have microphones and speakers. During the initial phase, the target device will send its public key to a controlling device by encoding the key using a fast codec that results in audio that is nonsensical to humans. This makes it so that attackers cannot listen in on the secret without the use of specialized equipment at close range. The receiving device will record this audio and decode it to retrieve the key. Afterwards the protocol will enter a second verification phase where each device will encode the information that was retrieved in the last phase using a slow codec. This codec is intended to be a more melodic tune for humans. Each device will play the slow codec audio and a user is supposed to determine that each device played matching tunes. This ensures that only the two desired devices were bootstrapped, giving feedback to the user that the process is complete. One method to perform a DoS attack is to play loud audio to disrupt the recording during the first phase. This method, however, should make attackers easily recognizable to a user. HAPADEP can also be prone to impersonation attacks, but that would require an attacker to have knowledge of the codec being used to decode and encode the public keys. Without this prior knowledge, any audio played back during the verification phase will be different from the original devices.

\subsubsection{Ultrasonic Pairing}
As an OOB channel, using an ultrasonic frequency [9] can present some complications for the security of bootstrapping. In order to securely pair two devices, they must be located in the same room without any third-party devices or physical obstacles in the vicinity. Successful pairing with ultrasonic frequencies is almost entirely dependent on location. If an attacker is in the same room, they would be able to mount a MiTM attack, impersonate another device, or eavesdrop on transmissions. The devices need to have clear line of sight with each other, otherwise it is possible that the sound waves will get distorted by physical obstacles or impersonated by an attacker. Ultrasound’s use as an OOB frequency for bootstrapping requires secondary safeguards to ensure security.


\subsection{Vibration}
\subsubsection{SYNCVIBE}
SYNCVIBE [20] is a method that uses vibration motors in smart devices for secure bootstrapping. The OOB channel is the physical vibration created by the motors and an accelerometer is used to decode the vibrations and receive the secret. In order for the devices to successfully pair with this method, they must have a primary channel to communicate in ala Wi-Fi or Bluetooth and must make physical contact with each other. This physical contact requirement is a way to notably reduce the possibility of MiTM attacks and eavesdroppers. The biggest impediments to vibration as a bootstrapping technique is the response time and lack of synchronization of smart device vibration motors. It is also possible for vibrations to be lost or altered during transit, corrupting the message.


\subsection{Light}
\subsubsection{LIRA/LIRA+}
The LIRA/LIRA+ [11] protocols utilize visible light as the auxiliary OOB channel to exchange bootstrapping information while Bluetooth, Wifi, or some other in-band radio frequency is used to verify that the messages were received. In order to utilize a visible light channel, devices must contain a photodiode and light sensor. A controller serves as the light source where the devices are placed on, in order to transmit cryptographic information by flashing a sequence of lights that other devices can decrypt to get the shared secret. From there they can use the secret to create a secure communication channel. The main vulnerabilities with this method of bootstrapping is that third parties may be able to view the flashing sequence and inject or interfere with the sequences by flashing their own light from a distance. Doing so, however, would make attackers obvious to users but this shows the success of this method is dependent on the environment that it is in.

\subsubsection{Group Message Authentication Protocol (GAP)}
GAP [16] is another light protocol that is designed around bootstrapping multiple devices at once.There must be an LED light and button on each device, and the devices must be within reach of the user. The devices will then be powered on and a controller device and coordinator will be set. Once there is a coordinator, a group message protocol is sent in order to verify identities over an unsecured radio frequency. The protocol will generate a common group authentication string on each device which will then be broadcasted over a partially authenticated visible light channel. The user will then manually verify that the common group string is the same among all devices by pushing a button on each verified device. One of the vulnerabilities with this protocol is that communication over a radio channel is susceptible to eavesdropping, MiTM attacks, replay attacks, and impersonation. Over visible light channels, attackers are unable to destroy messages that are sent out, but they are able to interfere and modify the messages by injecting their own light with the use of a laser. 


