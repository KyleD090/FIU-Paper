\documentclass[10pt,sigconf]{acmart}

\usepackage{booktabs} % For formal tables

\graphicspath{{figure/}{figures/}}

% Copyright
%\setcopyright{none}
%\setcopyright{acmcopyright}
%\setcopyright{acmlicensed}
\setcopyright{rightsretained}
%\setcopyright{usgov}
%\setcopyright{usgovmixed}
%\setcopyright{cagov}
%\setcopyright{cagovmixed}


% DOI
\acmDOI{10.475/123_4}

% ISBN
\acmISBN{123-4567-24-567/08/06}

%Conference
\acmConference[SHORTNAME'17]{ACM Long Conference Name conference}{July 1997}{City, State, Country} 
\acmYear{2017}
\copyrightyear{2017}

\acmPrice{15.00}


\begin{document}
\title{Can you join the network? Survey of Bootstrapping techniques in NDNoT}
% \titlenote{Produces the permission block, and copyright information}
% \subtitle{Extended Abstract}

% \author{Firstname Lastname}
% \authornote{Note}
% \orcid{1234-5678-9012}
% \affiliation{%
%   \institution{Affiliation}
%   \streetaddress{Address}
%   \city{City} 
%   \state{State} 
%   \postcode{Zipcode}
% }
% \email{email@domain.com}

% \author{Firstname Lastname}
% \orcid{1234-5678-9012}
% \affiliation{%
%   \institution{Affiliation}
%   \streetaddress{Address}
%   \city{City} 
%   \state{State} 
%   \postcode{Zipcode}
% }
% \email{email@domain.com}

% \author{Firstname Lastname}
% \orcid{1234-5678-9012}
% \affiliation{%
%   \institution{Affiliation}
% }
% \email{email@domain.com}

% \author{Firstname Lastname}
% \orcid{1234-5678-9012}
% \affiliation{%
%   \institution{Affiliation}
% }
% \email{email@domain.com}

% \author{Firstname Lastname}
% \orcid{1234-5678-9012}
% \affiliation{%
%   \institution{Affiliation}
% }
% \email{email@domain.com}


% The default list of authors is too long for headers}
% \renewcommand{\shortauthors}{F. Lastname et al.}


\begin{abstract}

  The rapid proliferation of sensors and their use in modern Internet of Things (IoT) environment has brought about a revolution in the way offices work.
  Researchers have identified the benefits that a data-centric approach like Named Data Networking can provide in such a scenario.
  The vast diversity of the sensors and the need for new services leads to the constant addition of newer devices making it an environment that is highly vulnerable to attacks. 
  In this poster, we try to provide an overview of various techniques that can be used for bootstrapping trust to potentially ~(a) ensure legitimate sensors / devices join the network and ~(b) the device identifies the correct network to join.
  An explorative effort in using a button-based approach to complete bootstrapping and the corresponding threat model and solution is discussed in this article.
  
\end{abstract}

% %
% % The code below should be generated by the tool at
% % http://dl.acm.org/ccs.cfm
% % Please copy and paste the code instead of the example below. 
% %
% \begin{CCSXML}
% <ccs2012>
%  <concept>
%   <concept_id>10010520.10010553.10010562</concept_id>
%   <concept_desc>Computer systems organization~Embedded systems</concept_desc>
%   <concept_significance>500</concept_significance>
%  </concept>
%  <concept>
%   <concept_id>10010520.10010575.10010755</concept_id>
%   <concept_desc>Computer systems organization~Redundancy</concept_desc>
%   <concept_significance>300</concept_significance>
%  </concept>
%  <concept>
%   <concept_id>10010520.10010553.10010554</concept_id>
%   <concept_desc>Computer systems organization~Robotics</concept_desc>
%   <concept_significance>100</concept_significance>
%  </concept>
%  <concept>
%   <concept_id>10003033.10003083.10003095</concept_id>
%   <concept_desc>Networks~Network reliability</concept_desc>
%   <concept_significance>100</concept_significance>
%  </concept>
% </ccs2012>  
% \end{CCSXML}

% \ccsdesc[500]{Computer systems organization~Embedded systems}
% \ccsdesc[300]{Computer systems organization~Redundancy}
% \ccsdesc{Computer systems organization~Robotics}
% \ccsdesc[100]{Networks~Network reliability}

% We no longer use \terms command
%\terms{Theory}

\keywords{ACM proceedings}


\maketitle

\section{Introduction}

Trust plays an integral part in network communications.
It effectively determines from whom and from which services or applications, users accept information or are willing to send information.
When a new device attempts to integrate itself into a new environment for the first time, it must go through a verification process that identifies the device to be legitimate.
Bootstrapping is the onboarding process through which an entity learns the presence of other entities in the network along with learning of the other entity’s current state.
It is the process that prevents the infiltration of malicious nodes into the network and helps in making sure that legitimate devices or entities are getting the necessary accesses or privileges.
Trust bootstrapping thus assists the requester in making the right choice of services 

\section{Background}


Trust plays an integral part in network communications.
It effectively determines from whom and from which services or applications, users accept information or are willing to send information.
When a new device attempts to integrate itself into a new environment for the first time, it must go through a verification process that identifies the device to be legitimate.
Bootstrapping is the onboarding process through which an entity learns the presence of other entities in the network along with learning of the other entity’s current state.
It is the process that prevents the infiltration of malicious nodes into the network and helps in making sure that legitimate devices or entities are getting the necessary accesses or privileges.
Trust bootstrapping thus assists the requester in making the right choice of services 

\section{Bootstrapping Taxonomy}

There has been a large amount of previous research on the various bootstrapping techniques for securely pairing devices. Out-of-band (OOB) channels are auxiliary to the main frequency of communication and are used as a means to bolster security. OOB channels are characterized by the uncommon or unused frequencies that the protocols used to communicate or share any cryptographic information they normally would not send on the 2.4 Ghz frequency range.

\subsection{Bluetooth Low Energy (BLE)}
\subsubsection{Device Provision Protocol (DPP)}
As a means for device onboarding, BLE is a viable alternative to Wifi in scenarios that do not require long range communication - not unlike an office. BLE lets a user send out a broadcast that states their intent to bootstrap with another device. The DPP requires two devices: a configurator that broadcasts intent and an enrollee within range that responds to a broadcast. When an enrollee does respond to a broadcast, the devices move to an auxiliary channel via BLE and exchange cryptographic information. One of the primary DPP characteristics is that the configurator sends out a request packet on the auxiliary channel and the enrollee must send back that same auxiliary channel request information to successfully onboard. One of the main vulnerabilities of using BLE pairing is its that broadcasts openly to any device within range. By broadcasting openly, the devices make themselves susceptible to Man in The Middle (MiTM) attacks. BLE itself does not have a proof of possession of the bootstrapping keys in the auxiliary channel. Any device within range can possibly eavesdrop on both device and obtain the bootstrapping keys.


\subsection{Buttons}
\subsubsection{Button Enabled Device Association (BEDA)}
The BEDA protocol is reliant on physical buttons on two devices. These buttons are integral to the protocol because it uses the duration of a pressed button which is used to generate a shared secret This method presents a few vulnerabilities, however. The protocol is vulnerable to replay attacks because hashes are not encrypted when they are sent out which lets an attacker pose as the sender by sending the hash to the original receiving device.

\subsection{Magnetometer Pairing}
Magnetometer pairing is technique of bootstrapping that was designed for smartphones by making use of their magnetometers and a WiFi connection. To initiate the process two devices must be tapped together which should trigger the protocol depending on the threshold that has been set. Once the process has been initiated the devices will go through a standard Diffie-Hellman (DH) Key Exchange. During this process the devices will record their magnetometer readings at the same time. In order to foil eavesdroppers, the protocol makes use of the Interlock protocol to bolster its security against passive attacks. The protocol is secure against MiTM attacks because the Interlock the attacker will only receive half of an encrypted message of which is has no key to decrypt. Magnetometer pairing makes use of the randomness of the strength of magnetic fields by incorporating the lack of temporal or spatial alignment between devices into


\bibliographystyle{acm}
\bibliography{refs} 

\end{document}
